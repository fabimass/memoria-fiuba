% Chapter 1

\chapter{Introducción general} % Main chapter title

\label{Chapter1} % For referencing the chapter elsewhere, use \ref{Chapter1} 
\label{IntroGeneral}

%----------------------------------------------------------------------------------------

% Define some commands to keep the formatting separated from the content 
\newcommand{\keyword}[1]{\textbf{#1}}
\newcommand{\tabhead}[1]{\textbf{#1}}
\newcommand{\code}[1]{\texttt{#1}}
\newcommand{\file}[1]{\texttt{\bfseries#1}}
\newcommand{\option}[1]{\texttt{\itshape#1}}
\newcommand{\grados}{$^{\circ}$}

%----------------------------------------------------------------------------------------

En este capítulo se introduce la problemática que motivó el presente trabajo, seguida de una breve descripción de la solución propuesta. 
A continuación, se expone el estado del arte de las tecnologías aplicadas. 
Finalmente, se detallan el alcance y los requerimientos necesarios para su implementación.

%----------------------------------------------------------------------------------------
\section{Introducción a la problemática}

En un entorno empresarial, la eficiencia en la búsqueda de información es crucial para la
productividad y el rendimiento de los empleados. Sin embargo, con la creciente cantidad de
datos y documentos disponibles, encontrar información específica de manera rápida y precisa
puede convertirse en un desafío.

A lo largo de mi experiencia en la empresa donde me desempeño, he observado cómo la abundancia 
de fuentes de información puede, paradójicamente, dificultar el trabajo. Existen múltiples 
repositorios de documentos, políticas y datos históricos, pero la falta de centralización y la 
dificultad para identificar la fuente correcta suelen traducirse en pérdidas de tiempo significativas. 
En muchas ocasiones, he dedicado más tiempo a la búsqueda de información que a la ejecución de las tareas 
en sí, lo que afecta tanto la productividad como la efectividad en la toma de decisiones.


%----------------------------------------------------------------------------------------
\section{Marco de la propuesta}

Un chatbot especializado ofrece una solución prometedora al permitir a los usuarios realizar 
consultas en lenguaje natural y obtener respuestas de manera instantánea. Mientras que otros sistemas 
de inteligencia artificial ampliamente conocidos y utilizados, como ChatGPT o Microsoft Copilot, 
destacan en su capacidad para generar respuestas generales basadas en un amplio conocimiento del 
lenguaje, el presente trabajo se distingue por su capacidad para trabajar con documentos altamente 
específicos (y potencialmente privados). Esto le permite ofrecer respuestas adaptadas al contexto 
interno de la organización, las cuales no podrían obtenerse mediante el uso de los chatbots de 
propósito general disponibles en el mercado.

En la figura \ref{fig:diagrama-bloques-basico} se presenta un diagrama de alto nivel de la solución.
En primer lugar, los usuarios interactúan con el chatbot a través de una interfaz gráfica, desde la
cual pueden realizar consultas sobre la información deseada. Estas consultas, procesadas mediante 
técnicas de lenguaje natural, permiten extraer la información más relevante de la fuente de documentos. 
Luego, un modelo de inteligencia artificial interpreta las consultas y genera respuestas adecuadas, 
proporcionando al usuario la información solicitada de manera precisa y contextualizada.

\begin{figure}[ht]
	\centering
	\includegraphics[scale=.3]{./Figures/diagrama_bloques_basico.png}
	\caption{Diagrama de alto nivel de la solución.}
	\label{fig:diagrama-bloques-basico}
\end{figure}

%----------------------------------------------------------------------------------------
\section{Estado del arte}

El desarrollo de chatbots y sistemas de recuperación de información ha avanzado considerablemente en los últimos años, 
impulsado por mejoras en el procesamiento de lenguaje natural (PLN) y el acceso a grandes volúmenes de datos. 
En este contexto, los chatbots especializados han surgido como soluciones destacadas para el acceso eficiente 
a información específica en distintos entornos, incluyendo el empresarial. A continuación, se presenta una revisión 
de las principales tecnologías y enfoques actuales que sustentan el desarrollo del presente trabajo.

Los chatbots modernos han evolucionado desde sistemas de reglas simples hasta modelos sofisticados capaces de 
mantener conversaciones complejas. Entre los primeros desarrollos de chatbots, como ELIZA \citep{paper:eliza} en 
la década de 1960, se empleaban reglas predefinidas que limitaban la interacción a una cantidad pequeña de 
respuestas posibles. Sin embargo, el uso de redes neuronales y el aprendizaje profundo en las últimas décadas 
ha transformado el campo de los chatbots, permitiendo la aparición de sistemas como Siri de Apple, Alexa de Amazon 
y Google Assistant \citep{article:voice-assistants}. Estos asistentes virtuales han popularizado el uso de interfaces 
de conversación en la vida cotidiana, siendo capaces de responder a preguntas comunes, realizar tareas administrativas 
y ofrecer asistencia en tiempo real.

Una tendencia reciente en el desarrollo de chatbots es la aplicación de modelos generativos de lenguaje, como GPT-3 y GPT-4 
de OpenAI \citep{paper:gpt}, BERT de Google \citep{paper:bert}, y LLAMA de Meta \citep{paper:llama}. Estos modelos, basados 
en arquitecturas de \textit{transformers} \citep{paper:transformers}, permiten una comprensión profunda del contexto y del 
significado en secuencias de palabras. Su capacidad de generar respuestas coherentes y bien estructuradas ha llevado al 
desarrollo de los tan populares chatbots modernos como ChatGPT \citep{website:chatgpt}, Microsoft Copilot \citep{website:copilot}
o Google Gemini \citep{website:gemini}, cuyas interfaces se observan en la figura \ref{fig:chatbots}.

\begin{figure}[ht]
	\centering
	\includegraphics[scale=.38]{./Figures/chatbots.png}
	\caption{ChatGPT, Gemini y Copilot, los chatbots más populares actualmente.}
	\label{fig:chatbots}
\end{figure}

\vspace{15mm}

Si bien los modelos generativos han alcanzado un alto grado de sofisticación, presentan algunas limitaciones importantes. 
En primer lugar, su conocimiento es en gran medida de propósito general, dado que han sido entrenados con grandes volúmenes 
de datos públicos y no específicos, lo cual limita su precisión cuando se requiere información particular de una organización. 
En segundo lugar, estos modelos tienden a ``inventar'' respuestas cuando no encuentran información relevante, fenómeno conocido 
como \textit{hallucinations} \citep{article:hallucinations}. En un contexto empresarial, esto puede provocar confusión o incluso 
proporcionar información errónea.

En la búsqueda de soluciones que combinen la capacidad de los modelos generativos con la precisión de la información propietaria, 
ha surgido el enfoque de generación aumentada por recuperación (RAG, por sus siglas en inglés). Este enfoque combina sistemas de 
recuperación de información con modelos de generación de texto, lo que permite que las respuestas no solo se basen en la capacidad 
generativa del modelo, sino también en una búsqueda previa en bases de datos o documentos específicos \citep{paper:rag-1} 
\citep{paper:rag-2}.

El presente trabajo se apoya en el estado del arte de los modelos de lenguaje y la técnica de RAG para crear una solución innovadora 
que mejora la productividad al centralizar y optimizar el acceso a la información relevante en el entorno laboral.

%----------------------------------------------------------------------------------------
\section{Motivación y alcance}

El propósito de este trabajo fue optimizar el proceso de búsqueda de información por parte de los empleados. Se buscó
proporcionar una herramienta eficaz que permita acceder rápidamente a los datos relevantes, que mejore la eficiencia y 
productividad en el entorno laboral.

Para ello, se realizaron las siguientes tareas:

\begin{itemize}
	\item Procesamiento de los documentos y posterior almacenamiento en una base de datos.
	\item Integración con un modelo lingüístico grande (LLM) que pueda entender las consultas de los usuarios y proporcionar respuestas precisas basadas en el contenido de los documentos ingestados.
	\item Diseño e implementación de una interfaz de usuario intuitiva y fácil de utilizar que permita a los empleados interactuar con el chatbot de manera eficiente.
	\item Desarrollo de un \textit{pipeline} de despliegue continuo que facilite la ingesta de nuevos documentos y la actualización de la aplicación.
  \item Evaluación del rendimiento del chatbot mediante pruebas exhaustivas con diferentes tipos de consultas.
\end{itemize}

Las siguientes actividades no formaron parte del alcance:

\begin{itemize}
	\item Despliegue del chatbot en un ambiente productivo.
	\item Entrenamiento continuo del chatbot en base a las consultas realizadas por los usuarios.
	\item Desarrollo de funcionalidades avanzadas de seguridad, tales como autenticación de usuarios o cifrado de datos.
\end{itemize}

%----------------------------------------------------------------------------------------
\section{Requerimientos}

A continuación se describen los principales requerimientos establecidos para cumplir
con el alcance propuesto: