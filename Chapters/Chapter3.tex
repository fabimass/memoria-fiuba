\chapter{Diseño e implementación} % Main chapter title

\label{Chapter3} % Change X to a consecutive number; for referencing this chapter elsewhere, use \ref{ChapterX}

\definecolor{mygreen}{rgb}{0,0.6,0}
\definecolor{mygray}{rgb}{0.5,0.5,0.5}
\definecolor{mymauve}{rgb}{0.58,0,0.82}

%%%%%%%%%%%%%%%%%%%%%%%%%%%%%%%%%%%%%%%%%%%%%%%%%%%%%%%%%%%%%%%%%%%%%%%%%%%%%
% parámetros para configurar el formato del código en los entornos lstlisting
%%%%%%%%%%%%%%%%%%%%%%%%%%%%%%%%%%%%%%%%%%%%%%%%%%%%%%%%%%%%%%%%%%%%%%%%%%%%%
\lstset{ %
  backgroundcolor=\color{white},   % choose the background color; you must add \usepackage{color} or \usepackage{xcolor}
  basicstyle=\footnotesize,        % the size of the fonts that are used for the code
  breakatwhitespace=false,         % sets if automatic breaks should only happen at whitespace
  breaklines=true,                 % sets automatic line breaking
  captionpos=b,                    % sets the caption-position to bottom
  commentstyle=\color{mygreen},    % comment style
  deletekeywords={...},            % if you want to delete keywords from the given language
  %escapeinside={\%*}{*)},         % if you want to add LaTeX within your code
  %extendedchars=true,             % lets you use non-ASCII characters; for 8-bits encodings only, does not work with UTF-8
  %frame=single,	                 % adds a frame around the code
  keepspaces=true,                 % keeps spaces in text, useful for keeping indentation of code (possibly needs columns=flexible)
  keywordstyle=\color{blue},       % keyword style
  language=[ANSI]C,                % the language of the code
  %otherkeywords={*,...},          % if you want to add more keywords to the set
  numbers=left,                    % where to put the line-numbers; possible values are (none, left, right)
  numbersep=5pt,                   % how far the line-numbers are from the code
  numberstyle=\tiny\color{mygray}, % the style that is used for the line-numbers
  rulecolor=\color{black},         % if not set, the frame-color may be changed on line-breaks within not-black text (e.g. comments (green here))
  showspaces=false,                % show spaces everywhere adding particular underscores; it overrides 'showstringspaces'
  showstringspaces=false,          % underline spaces within strings only
  showtabs=false,                  % show tabs within strings adding particular underscores
  stepnumber=1,                    % the step between two line-numbers. If it's 1, each line will be numbered
  stringstyle=\color{mymauve},     % string literal style
  tabsize=2,	                     % sets default tabsize to 2 spaces
  title=\lstname,                  % show the filename of files included with \lstinputlisting; also try caption instead of title
  morecomment=[s]{/*}{*/}
}


%----------------------------------------------------------------------------------------

Este capítulo describe el diseño y la implementación de cada uno de los componentes que conforman el sistema.
Se explican las decisiones de diseño, los flujos de trabajo y los aspectos técnicos involucrados en la construcción de cada módulo principal.

%---------------------------------------------------------------------------------------
\section{Arquitectura del sistema}

El sistema está diseñado para permitir la implementación y operación de un chatbot mediante un enfoque de recuperación aumentada por generación. 
Se implementó una arquitectura modular, basada en servicios en la nube, que facilita la escalabilidad y el mantenimiento del chatbot, y permite 
una actualización ágil de los componentes y una experiencia optimizada para los usuarios finales. En la figura \ref{fig:architecture} se ilustra
el diagrama de arquitectura, donde se observan la totalidad de los componentes que conforman el sistema.

En primer lugar, los usuarios interactúan con el chatbot a través de una interfaz gráfica (desarrollada con NextUI y deployada en Azure Static Web App), 
donde consultan por la información deseada. Estas consultas son luego transferidas al servidor (Azure App Service) a través de una API desarrollada 
con la librería FastAPI. Una vez allí, se realiza un proceso de búsqueda por similitud que toma la solicitud del usuario y la compara con la información
disponible en la base de datos (Azure AI Search), con el objetivo de identificar aquellos fragmentos más relevantes. Previamente, un administrador
debe haber cargado aquellos documentos que conforman la base de conocimiento del chatbot en un repositorio de GitHub, tras lo cual se ejecuta una 
automatización que los procesa y los envía a la base de datos.

Finalmente, la consulta del usuario y los fragmentos relevantes se envían al modelo LLM (desplegado en el servicio Azure OpenAI), que genera una 
respuesta contextualizada, que es devuelta a la interfaz gráfica para ser presentada al usuario.

Además, se incluye una pequeña base de datos adicional (Azure Table Storage) que se utiliza para guardar el feedback de los usuarios, 
para así obtener métricas del desempeño del chatbot.  

\vspace{30mm}

\begin{figure}[ht]
	\centering
	\includegraphics[scale=.55]{./Figures/arquitectura.png}
	\caption{Diagrama de arquitectura del sistema.}
	\label{fig:architecture}
\end{figure}

\vspace{8mm}

%---------------------------------------------------------------------------------------
\section{Configuración de la infraestructura en la nube}

Para garantizar el despliegue, disponibilidad y escalabilidad del sistema, se configuró una infraestructura en la nube basada en Microsoft Azure. 
A continuación, se describen los pasos principales para la configuración de cada recurso empleado en el proyecto.

\subsection{OpenAI Service}

El servicio de Azure OpenAI proporciona acceso a los potentes modelos de lenguaje de OpenAI, incluidos los más recientes. 
Estos modelos pueden adaptarse fácilmente a tareas específicas, como en este caso es la generación de contenido.

En la tabla \ref{tab:config-openai} se presenta un resumen de las configuraciones realizadas. 
Se seleccionó \textit{East US} como región, junto con el modelo de precios \textit{Standard S0}, 
adecuado para balancear el costo y el rendimiento del sistema.

Como medida de seguridad, se configuraron reglas de red para restringir el acceso únicamente al rango 
de direcciones IP del backend alojado en Azure App Service. Esto asegura que solo las solicitudes provenientes de la 
aplicación puedan acceder al modelo de lenguaje, lo que protege el servicio de accesos no autorizados.

Una vez desplegado el recurso, fue necesario seleccionar y desplegar los modelos específicos requeridos por el chatbot. 
En este caso, se optó por los siguientes modelos:

\begin{itemize}
	\item \textit{GPT-4o} como modelo de lenguaje para generación de respuestas. Este modelo está diseñado para brindar velocidad 
  y eficiencia, iguala la inteligencia de su antecesor \textit{GPT-4 Turbo}, y es notablemente más eficiente al entregar texto al 
  doble de velocidad y a la mitad del costo. Además, exhibe el rendimiento más alto en idiomas distintos del inglés en comparación 
  con los modelos de OpenAI anteriores.
	\item \textit{Ada-002} para el cálculo de \textit{embeddings} de texto. Este modelo supera a todos los modelos de \textit{embeddings} 
  anteriores en tareas de búsqueda de texto y similitud de oraciones.
\end{itemize}

Adicionalmente, es fundamental obtener el \textit{endpoint} y la \textit{key} del servicio, valores que se utilizan luego para la comunicación programática 
con Azure OpenAI. Estos datos se almacenan en el backend como variables de entorno para que el sistema pueda acceder al servicio de manera segura.

\begin{table}[h]
	\centering
	\caption[Configuración de Azure OpenAI]{Configuración de Azure OpenAI}
	\begin{tabular}{l l}    
		\toprule
		\textbf{Configuración} 	 & \textbf{Detalles} 	     \\
		\midrule
		Región                   &	East US 				 \\		
		Nivel de precios         & Standard S0				 \\
		Reglas de firewall       & Rango IP de App Service   \\
        Modelos desplegados	     & - gpt-4o				     \\
            	                 & - text-embeddings-ada-002 \\
        Credenciales	         & Endpoint y key 		     \\
		\bottomrule
		\hline
	\end{tabular}
	\label{tab:config-openai}
\end{table}

\subsection{AI Search}

Azure AI Search es un servicio de búsqueda en la nube con capacidades de inteligencia artificial integradas 
que enriquecen todo tipo de información con el fin de identificar y explorar fácilmente contenido relevante a escala.

La tabla \ref{tab:config-ai-search} presenta un resumen de las configuraciones realizadas. Se seleccionó la región de \textit{East US} y un nivel de precios 
\textit{Standard}, que ofrece hasta 50 índices para el almacenamiento y la búsqueda de documentos. En cuanto a la escala, el servicio se configuró para 
utilizar una sola unidad de búsqueda, lo cual es adecuado en principio para el volumen de consultas esperado y los requisitos de rendimiento.

Al igual que con el servicio de Azure OpenAI, se configuraron reglas de red restrictivas para mejorar la seguridad del servicio, al permitir 
únicamente el acceso desde el rango de direcciones IP asociado al backend. 

Aquí también es fundamental obtener la \textit{key} y el \textit{endpoint} del servicio para integrarlos en las variables de entorno del backend.

Es importante destacar que no es necesario crear un índice manualmente en esta etapa, ya que este será generado automáticamente como parte del pipeline de despliegue.

\begin{table}[h]
	\centering
	\caption[Configuración de Azure AI Search]{Configuración de Azure AI Search}
	\begin{tabular}{l l}    
		\toprule
		\textbf{Configuración} 	 & \textbf{Detalles} 	   \\
		\midrule
		Región                   &	East US 			   \\		
		Nivel de precios         & Standard				   \\
		Reglas de firewall       & Rango IP de App Service \\
    	Credenciales	         & Endpoint y key 		   \\
		\bottomrule
		\hline
	\end{tabular}
	\label{tab:config-ai-search}
\end{table}

\subsection{App Service}

Este servicio ofrece una plataforma completamente administrada donde es posible alojar una aplicación avanzada en la nube 
sin necesidad de manejar la infraestructura asociada. En este caso se utilizó para desplegar el backend del chatbot.

La tabla \ref{tab:config-app-service} resume la configuración principal del servicio. En primer lugar, se seleccionó un App Service Plan de categoría \textit{Basic B3}, que proporciona un poder de procesamiento de 
4 vCPU, 7 GB de memoria RAM, 10 GB de almacenamiento remoto, y permite escalar hasta 3 instancias en caso de ser necesario. 
Como entorno de ejecución, se configuró Python 3.10, mientras que como región se optó una vez más por \textit{East US}. Con el 
fin de optimizar los costos, se deshabilita la opción de redundancia zonal.

Para que la aplicación funcione correctamente, es esencial configurar un conjunto variables de entorno, que se listan en la tabla \ref{tab:config-env}. Estas 
variables incluyen las claves y los puntos de conexión de los servicios de Azure que interactúan con el backend. 

\begin{table}[h]
	\centering
	\caption[Variables de entorno]{Variables de entorno}
	\begin{tabular}{l l}    
		\toprule
		\textbf{Variable de entorno}  & \textbf{Descripción} 	                          \\
		\midrule
		openai\_api\_key              &	Clave de acceso del servicio de Azure OpenAI 	  \\		
		openai\_endpoint              & Endpoint del servicio de Azure OpenAI			  \\
		search\_key                   & Clave de acceso del servicio de Azure AI Search   \\
		search\_endpoint	          & Endpoint del servicio de Azure AI Search		  \\
        storage\_connection\_string   & Connection string de la cuenta de almacenamiento  \\
		db\_index	                  & Índice de la base de datos a utilizar		      \\
		\bottomrule
		\hline
	\end{tabular}
	\label{tab:config-env}
\end{table}

Para asegurar un monitoreo efectivo de la aplicación, se habilitó la integración con Application Insights, lo que permite un 
seguimiento detallado de las métricas de rendimiento y de uso. Adicionalmente, se configuró la opción de \textit{application logging}, 
de modo que los logs de la aplicación fueran visualizados directamente desde la terminal del App Service, facilitando la depuración 
y la supervisión continua.

\begin{table}[h]
	\centering
	\caption[Configuración de Azure App Service]{Configuración de Azure App Service}
	\begin{tabular}{l l}    
		\toprule
		\textbf{Configuración} & \textbf{Detalles} 	\\
		\midrule
		App Service Plan       & Basic B3           \\
		Runtime                & Python 3.10        \\
		Región                 & East US 			\\		
		Zone redundancy        & Deshabilitada		\\
		Application Insights   & Habilitado         \\
		Application logging	   & Filesystem			\\
		\bottomrule
		\hline
	\end{tabular}
	\label{tab:config-app-service}
\end{table}

\subsection{Static Web App}

Al utilizar este servicio, el contenido estático como HTML, CSS, JavaScript e imágenes, se distribuye globalmente desde diversos puntos 
alrededor del mundo a diferencia de un servidor web tradicional, por lo que los archivos se encuentran físicamente más cerca de los usuarios finales,
sin importar su ubicación.

La tabla \ref{tab:config-static-webapp} resume los aspectos clave de la Static Web App. Este recurso es sencillo de configurar, ya que solo requiere seleccionar un modelo de precios. En este caso, se optó por 
la versión \textit{Standard}. A diferencia de otros servicios en Azure, las Static Web Apps se despliegan globalmente, por lo que no es necesario especificar una región. 
Además, no se requieren variables de entorno para este recurso, ya que todas las conexiones con otros servicios son manejadas exclusivamente 
por el backend, simplificando la configuración.

Es importante obtener el \textit{deployment token} asociado al recurso, que será necesario más adelante para configurar el pipeline de despliegue automático, 
permitiendo la autenticación y el despliegue seguro desde GitHub.

\begin{table}[h]
	\centering
	\caption[Configuración de Azure App Service]{Configuración de Azure App Service}
	\begin{tabular}{l l}    
		\toprule
		\textbf{Configuración} 	& \textbf{Detalles} \\
		\midrule
		Nivel de precios        & Standard          \\
		Región                  & Global 			\\		
		Variables de entorno    & No aplica			\\
		Credenciales            & Deployment token  \\
		\bottomrule
		\hline
	\end{tabular}
	\label{tab:config-static-webapp}
\end{table}

\subsection{Table Storage}

Azure Table Storage es un servicio que almacena datos estructurados no relacionales (también conocidos como datos NoSQL) en la nube, que almacena claves/atributos 
con un diseño sin esquemas. En el contexto de este trabajo se utilizó para almacenar el feedback de los usuarios acerca de las respuestas provistas por el chatbot.

La tabla \ref{tab:config-table-storage} resume la configuración principal de este recurso. Primeramente, fue necesario desplegar una Storage Account en la región \textit{East US}, para así mantener la consistencia con los demás 
recursos del sistema. Se seleccionó una performance de tipo \textit{Standard}, adecuada para las necesidades de la aplicación.

En cuanto a la redundancia de los datos, se eligió la oferta de redundancia local, una alternativa rentable que asegura que los datos 
estén replicados dentro de un único centro de datos en la misma región, reduciendo costos sin comprometer la disponibilidad básica.

Una vez que la cuenta de almacenamiento fue desplegada, se procedió a crear la tabla que almacenará los resultados 
recopilados a través del sistema. Es importante obtener la \textit{connection string} asociada al recurso, ya que será 
necesaria para conectar la aplicación con la tabla y permitir el acceso programático. 

\begin{table}[h]
	\centering
	\caption[Configuración de Azure Table Storage]{Configuración de Azure Table Storage}
	\begin{tabular}{l l}    
		\toprule
		\textbf{Configuración} 	 & \textbf{Detalles} 	      \\
		\midrule
		Región                   &	East US 				  \\		
		Performance	             &  Standard				  \\
		Redundancia	             &  Locally-redundant storage \\
		Credenciales             &  Connection string         \\
		Nombre de la tabla       &	FeedbackTable             \\
		\bottomrule
		\hline
	\end{tabular}
	\label{tab:config-table-storage}
\end{table}

%---------------------------------------------------------------------------------------
\section{Procesamiento de los documentos}

%---------------------------------------------------------------------------------------
\section{Lógica de comunicación entre el usuario y el modelo}

%---------------------------------------------------------------------------------------
\section{API}

%---------------------------------------------------------------------------------------
\section{Interfaz de usuario}

%---------------------------------------------------------------------------------------
\section{Pipelines de despliegue automático}